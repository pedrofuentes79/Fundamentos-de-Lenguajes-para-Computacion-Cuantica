\documentclass[12pt,a4paper]{article}

% Packages for mathematical typesetting
\usepackage{amsmath}
\usepackage{amssymb}
\usepackage{amsthm}

% Packages for quantum circuits
\usepackage{xypic}
\usepackage{qcircuit}

% Package for ket notation
\usepackage{braket}

% Title and author
\title{Fundamentos de Lenguajes para Computación Cuántica}
\author{Pedro Fuentes Urfeig, Matías Mesch}


\begin{document}


\section{Ejercicio II.5}

Demostrar el teorema 4.12. En \textbf{Set}, los epimorfismos son las funciones sobreyectivas.
Osea, toda flecha $f: A \to B$ cumple que $\forall b \in B, \exists a \in A$ tal que $f(a) = b$.

Ida: f es un epimorfismo $\implies$ f es sobreyectiva. 
Vamos a demostrarlo por el contrarrecíproco. Es decir f no es sobreyectiva $\implies$ f no es un epimorfismo.
Entonces, quiero buscar $g: B \to C, h: B \to C$ con $g \neq h$ pero $g \circ f = h \circ f$.

La idea es que $g(b_0) \neq h(b_0)$ para algún $b_0 \in B$, pero que tengan el mismo comportamiento sobre la imagen de $f$.

Por simpleza tomemos $g(b) = 0$ y $h(b) = \begin{cases} 0 & \text{si } b \neq b_0 \\ 1 & \text{si } b = b_0 \end{cases}$

De este modo, $g(b_0) \neq h(b_0)$, por ende $g \neq h$.

Veamos ahora $g \circ f = h \circ f \iff (g \circ f)(a) = (h \circ f)(a) \forall a \in A$.
Vemos que $f(a) \neq b_0 \forall a \in A$. Luego,
\begin{itemize}
    \item $(g \circ f)(a) = g(f(a)) = 0$, pues $g$ es constante y $f(a) \in Dom(g)$
    \item $(h \circ f)(a) = h(f(a)) = 0$, pues $f(a) \neq b_0$ por hipótesis.
\end{itemize}

Luego, vemos que $g \circ f = h \circ f$ para todo $a \in A$, pero no vale que $g = h$, por ende $f$ no es un epimorfismo.


Vuelta: f es sobreyectiva $\implies$ f es un epimorfismo.
Quiero ver que $g \circ f = h \circ f \implies g = h$, para $g: B \to C, h: B \to C$.
Como asumo que $f$ es sobreyectiva, sé que toda su imagen $B$ está alcanzada por un $f(a)$, con $a \in A$.

Vamos a demostrarlo por absurdo.
Asumamos que $g(b_1) \neq h(b_1)$ para algún $b_1 \in B$.
Luego, como $b_1 \in B$, sabemos que existe un $a_1 \in A$ tal que $f(a_1) = b_1$.

Entonces, como sabemos que $g \circ f = h \circ f$, tenemos que:

$$(g \circ f)(a_1) = (h \circ f)(a_1)$$
$$g(f(a_1)) = h(f(a_1))$$
$$g(b_1) = h(b_1)$$

Absurdo! Pues habíamos asumido que $g(b_1) \neq h(b_1)$.
Luego, $g = h$.
Entonces, $f$ es un epimorfismo.

\qed

\section{Ejercicio III.15}

Dar un circuito que genere el estado $\frac{1}{\sqrt{2}} (\ket{000} + \ket{111})$ a partir de la entrada $\ket{000}$.

Un circuito que cumple es el siguiente:
\[
\Qcircuit @C=1em @R=.7em {
  & \lstick{|0\rangle} & \gate{H} & \ctrl{1} & \qw      & \qw \\
  & \lstick{|0\rangle} & \qw      & \targ    & \ctrl{1} & \qw \\
  & \lstick{|0\rangle} & \qw      & \qw      & \targ    & \qw \\
}
\]

Veamos que esto es correcto. 
Comenzamos con el estado $\ket{\psi_0} = \ket{000}$, que es lo mismo que $\ket{0} \otimes \ket{0} \otimes \ket{0}$.
Vemos que al aplicar Hadamard al primer qubit obtenemos lo siguiente:
$$\ket{\psi_1} = \frac{1}{\sqrt{2}} (\ket{0} + \ket{1}) \otimes \ket{0} \otimes \ket{0}$$

Luego, al aplicar el CNOT con el control en el primer qubit y el target en el segundo, tenemos lo siguiente:
$$\ket{\psi_2} = \frac{1}{\sqrt{2}} (\ket{0} + \ket{1}) \otimes \frac{1}{\sqrt{2}} (\ket{0} + \ket{1}) \otimes \ket{0}$$

Esto es porque el control es el primer qubit. Entonces, si el primer qubit es 0, no hace nada (entonces queda en 0 el segundo). Pero si el primero es 1, se cambia el segundo qubit a 1.

Luego, al aplicar el CNOT con el control en el segundo qubit y el target en el tercero, sucede lo mismo pero en el tercer qubit.

$$\ket{\psi_3} = \frac{1}{\sqrt{2}} (\ket{0} + \ket{1}) \otimes \frac{1}{\sqrt{2}} (\ket{0} + \ket{1}) \otimes \frac{1}{\sqrt{2}} (\ket{0} + \ket{1}) $$

Que es lo mismo que $\frac{1}{\sqrt{2}} (\ket{000} + \ket{111})$.



\end{document}

\documentclass[12pt,a4paper]{article}

% Packages for mathematical typesetting
\usepackage{amsmath}
\usepackage{amssymb}
\usepackage{amsthm}

% Packages for quantum circuits
\usepackage{xypic}
\usepackage{qcircuit}

% Package for ket notation
\usepackage{braket}

% Title and author
\title{Fundamentos de Lenguajes para Computación Cuántica}
\author{Pedro Fuentes Urfeig, Matías Mesch}


\begin{document}

\maketitle

\section{Ejercicio III.15}

Dar un circuito que genere el estado $\frac{1}{\sqrt{2}} (\ket{000} + \ket{111})$ a partir de la entrada $\ket{000}$.

Un circuito que cumple es el siguiente:
\[
\Qcircuit @C=1em @R=.7em {
  & \lstick{|0\rangle} & \gate{H} & \ctrl{1} & \qw      & \qw \\
  & \lstick{|0\rangle} & \qw      & \targ    & \ctrl{1} & \qw \\
  & \lstick{|0\rangle} & \qw      & \qw      & \targ    & \qw \\
}
\]

Veamos que esto es correcto. 
Comenzamos con el estado $\ket{\psi_0} = \ket{000}$, que es lo mismo que $\ket{0} \otimes \ket{0} \otimes \ket{0}$.
Vemos que al aplicar Hadamard al primer qubit obtenemos lo siguiente:
$$\ket{\psi_1} = \frac{1}{\sqrt{2}} (\ket{0} + \ket{1}) \otimes \ket{0} \otimes \ket{0}$$

Luego, al aplicar el CNOT con el control en el primer qubit y el target en el segundo, tenemos lo siguiente:
$$\ket{\psi_2} = \frac{1}{\sqrt{2}} (\ket{0} + \ket{1}) \otimes \frac{1}{\sqrt{2}} (\ket{0} + \ket{1}) \otimes \ket{0}$$

Esto es porque el control es el primer qubit. Entonces, si el primer qubit es 0, no hace nada (entonces queda en 0 el segundo). Pero si el primero es 1, se cambia el segundo qubit a 1.

Luego, al aplicar el CNOT con el control en el segundo qubit y el target en el tercero, sucede lo mismo pero en el tercer qubit.

$$\ket{\psi_3} = \frac{1}{\sqrt{2}} (\ket{0} + \ket{1}) \otimes \frac{1}{\sqrt{2}} (\ket{0} + \ket{1}) \otimes \frac{1}{\sqrt{2}} (\ket{0} + \ket{1}) $$

Que es lo mismo que $\frac{1}{\sqrt{2}} (\ket{000} + \ket{111})$.



\end{document}
